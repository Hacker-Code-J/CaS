\section*{List of Symbols}

%The Lightweight Encryption Algorithm (LEA) is a block cipher developed in South Korea that is specifically designed for efficient implementation in both software and hardware. It's well-suited for resource-constrained environments due to its simplicity and low resource requirements. Here's a detailed breakdown of its technical aspects:
%
%Block and Key Size: LEA operates on a block size of 128 bits and supports key sizes of 128, 192, and 256 bits.
%
%Structure: The algorithm uses a generalized Feistel structure. It consists of a series of simple operations, such as XOR, addition modulo 
%2
%32
%2 
%32
%, and left/right rotations.
%
%Rounds: The number of rounds in LEA depends on the key size. For 128-bit keys, it uses 24 rounds; for 192-bit keys, 28 rounds; and for 256-bit keys, 32 rounds.
%
%Key Schedule: LEA employs a non-linear key schedule algorithm to generate round keys. The round keys are derived from the original key using simple arithmetic and bitwise operations.
%
%Round Function: Each round of LEA involves a few basic operations. The plaintext block is divided into four 32-bit words. These words are then processed through a mix of addition, XOR, and rotation operations, which vary slightly per round.
%
%Efficiency: The algorithm's simplicity allows for fast execution in software, particularly on platforms with limited processing power. It is also amenable to compact and efficient hardware implementations.
%
%Security: LEA has been analyzed for cryptographic security and has shown resilience against various forms of cryptanalysis, including differential and linear cryptanalysis.
%
%\[
%\begin{aligned}
%	&\textbf{LEA Round Function:} \\
%	&R_{i+1,1} = (R_{i,1} \oplus T_{i,1}) \lll 9; \\
%	&R_{i+1,2} = R_{i,2} \oplus T_{i,2}; \\
%	&R_{i+1,3} = R_{i,3} \lll 5; \\
%	&R_{i+1,4} = R_{i,4} \oplus T_{i,3}; \\
%	&\text{where } \lll \text{ denotes left rotation, } \oplus \text{ is XOR, and } T_{i,j} \text{ are the round-dependent keys.}
%\end{aligned}
%\]
%
%\begin{tikzpicture}[
%	node distance=1.5cm,
%	roundnode/.style={circle, draw=black, fill=gray!10, thick, minimum size=7mm},
%	squarednode/.style={rectangle, draw=black, thick, minimum size=5mm},
%	]
%	
%	% Nodes
%	\node[squarednode]      (R1)                              {R1};
%	\node[squarednode]      (R2)       [below=of R1]          {R2};
%	\node[squarednode]      (R3)       [below=of R2]          {R3};
%	\node[squarednode]      (R4)       [below=of R3]          {R4};
%	\node[roundnode]        (XOR1)     [right=of R1]          {XOR};
%	\node[roundnode]        (XOR2)     [right=of R4]          {XOR};
%	\node[squarednode]      (T1)       [above=of XOR1]        {T1};
%	\node[squarednode]      (T2)       [right=of R2]          {T2};
%	\node[squarednode]      (T3)       [below=of XOR2]        {T3};
%	\node[squarednode]      (Out1)     [right=of XOR1]        {Out1};
%	\node[squarednode]      (Out2)     [right=of T2]          {Out2};
%	\node[squarednode]      (Out3)     [right=of R3]          {Out3};
%	\node[squarednode]      (Out4)     [right=of XOR2]        {Out4};
%	
%	% Lines
%	\draw[->] (R1.east) -- (XOR1.west);
%	\draw[->] (R2.east) -- (T2.west);
%	\draw[->] (R3.east) -- (Out3.west);
%	\draw[->] (R4.east) -- (XOR2.west);
%	\draw[->] (T1.south) -- (XOR1.north);
%	\draw[->] (T3.north) -- (XOR2.south);
%	\draw[->] (XOR1.east) -- (Out1.west);
%	\draw[->] (T2.east) -- (Out2.west);
%	\draw[->] (XOR2.east) -- (Out4.west);
%	
%\end{tikzpicture}

\newpage